% multiple1902 <multiple1902@gmail.com>
% jiang-hk.tex
% Copyright 2011~2012, multiple1902 (Weisi Dai)
% https://code.google.com/p/xjtuthesis/
% 
% It is strongly recommended that you read documentations located at
%   http://code.google.com/p/xjtuthesis/wiki/Landing?tm=6
% in advance of your compilation if you have not read them before.
%
% This work may be distributed and/or modified under the
% conditions of the LaTeX Project Public License, either version 1.3
% of this license or (at your option) any later version.
% The latest version of this license is in
%   http://www.latex-project.org/lppl.txt
% and version 1.3 or later is part of all distributions of LaTeX
% version 2005/12/01 or later.
%
% This work has the LPPL maintenance status `maintained'.
% 
% The Current Maintainer of this work is Weisi Dai.
%
\chapter{江学长怒斥香港记者}

(2000年10月27日,北京中南海)

港女记:江主席,你觉得董先生连任,好不好啊?

江学长:好喔!

港女记:中央也支持他哈?

江学长:当然啦!

港女记:那为什么这么早就决定了,而不考虑别的人选了?

港女记:欧盟呢,最近发表了一个报告说呢,北京会透过一些渠道去影响、干预香港的法治。你对这个看法有什么回应呢?

江学长:没听到过!

港女记:这是彭定康说的。

江学长:彭定康说的就是真的了?你们媒体千万要注意了,不要见着风是得雨,接到这些消息,媒体本身也要判断,明白这意思吗?假使这些完全无中生有的东西,你再帮他说一遍,你等于……你也有责任吧?

港女记:现在那么早,你们就说支持董先生,会不会给人一种感觉,就是内定呀、是钦点呀董先生呀?

江学长:没有,没有任何的意思!还是按照香港的、按照基本法、按照选举的法,去产生……

港女记:但是你们能不能……

江学长:刚才你们问我呀,我可回答你一句「无可奉告」。但是你们又不高兴,那怎么办?!

港女记:可……

江学长:我讲的意思,不是钦点他当下一任。你问我支持不支持,我说支持。我就明确给你、告诉你。

港女记:江主席……

江学长:但是你们吧,你们……我感觉你们新闻界,还要学习一个……你们非常熟悉西方的那一套的理论,你们毕竟还too young!明白这意思吗?!我告诉你们,我是身经百战了,见得多了。西方的哪个国家我没去过?你们要知道,美国的华莱士,那比你们不知要高到哪里去了!嗯,我跟他谈笑风生。就是说媒体呀,还是要提高自己的知识水平。识得不识得啊?

江学长:唉,我也替你们着急呀,真的。你们就……你们有一个好,全世界跑到什么地方,你们比其他的西方记者,跑得还快!但是问来问去的问题呀,都too simple,sometimes na\"ive。懂了没啊?

港女记:可江主席……哦……

港女记:能不能说一下,为什么要支持董先生?

江学长:我很抱歉,我今天是作为一个长者,跟你们讲的。我不是新闻工作者,但是我见得太多了……我……有这个必要告诉你们一点,人生的经验。

江学长:刚才我很想啊,就我每次碰到你们,我就想到中国有句话叫「闷声发大财」。

港男记:那叫什么话?一句……

江学长:这是最好的!但是我想我见到你们这样热情呀,一句话不说也不好。所以刚才你一定要……在宣传上将来如果你们报道上敢有偏差,你们要负责!我没有说要钦定,没有任何这个意思。但是你问,一定要问我,对董先生支持不支持。我们不支持他?他现在当特首,我们怎么能不支持特首?!对不对?

港女记:但是如果说连任呢?

江学长:哎,连任也按照香港的法律呀,对不对?要、要按着香港的……当然我们的决定权,也是很重要的!香港特区、特别行政区是属于中国、中华人民共和国中央人民政{}府,啊!到那个时候,我们会表态的!


港女记:江主席……

江学长:但是你们吧,你们……我感觉你们新闻界,还要学习一个……你们非常熟悉西方的那一套的理论,你们毕竟还too young!明白这意思吗?!我告诉你们,我是身经百战了,见得多了。西方的哪个国家我没去过?你们要知道,美国的华莱士,那比你们不知要高到哪里去了!嗯,我跟他谈笑风生。就是说媒体呀,还是要提高自己的知识水平。识得不识得啊?

江学长:唉,我也替你们着急呀,真的。你们就……你们有一个好,全世界跑到什么地方,你们比其他的西方记者,跑得还快!但是问来问去的问题呀,都too simple,sometimes na\"ive。懂了没啊?

港女记:可江主席……哦……

港女记:能不能说一下,为什么要支持董先生?

江学长:我很抱歉,我今天是作为一个长者,跟你们讲的。我不是新闻工作者,但是我见得太多了……我……有这个必要告诉你们一点,人生的经验。

江学长:刚才我很想啊,就我每次碰到你们,我就想到中国有句话叫「闷声发大财」。

港男记:那叫什么话?一句……

江学长:这是最好的!但是我想我见到你们这样热情呀,一句话不说也不好。所以刚才你一定要……在宣传上将来如果你们报道上敢有偏差,你们要负责!我没有说要钦定,没有任何这个意思。但是你问,一定要问我,对董先生支持不支持。我们不支持他?他现在当特首,我们怎么能不支持特首?!对不对?

港女记:但是如果说连任呢?

江学长:哎,连任也按照香港的法律呀,对不对?要、要按着香港的……当然我们的决定权,也是很重要的!香港特区、特别行政区是属于中国、中华人民共和国中央人民政{}府,啊!到那个时候,我们会表态的!


港女记:但是呢……

江学长:明白这个意思吗?你们呀,不要想喜欢,啊,弄那么个大新闻,说现在已经钦定了,就把我批判一番!

问:但是……

江:你们啊,Na\"ive ……I'm angry,我跟你们讲,你们这样子不行的。我今天算得罪了你们一下。
