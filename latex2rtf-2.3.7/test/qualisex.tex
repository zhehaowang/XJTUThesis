\documentclass[twoside,12pt]{report} 
\usepackage[T1]{fontenc}
\usepackage[cp850]{inputenc}
\usepackage[french]{babel}
\global\tolerance10000 \global\pretolerance500 

\title{\Large\bf CHARTE de QUALIT\'E\\Sites sportifs d'escalade \\
       {\normalsize\bf(Projet)}
}
\author{\large\rm G\'erard \sc Decorps\\\large\rm Daniel \sc Taupin \\\large\rm
Jo\"el \sc Thomine}
\begin{document}
 \maketitle\tableofcontents\setcounter{secnumdepth}{3} \chapter{Remarques
pr\'eambulatoires}

La demande, �manant de minist\`eres par l'interm�diaire du Conseil
Sup�rieur des Sport de Montagne, de <<~chartes de qualit�~>> des {\sl
sites\/} et des {\sl topos-guides\/} d'escalade appellent deux remarques de la
part des r�dacteurs de ces {\sl chartes\/}~:

\begin{enumerate}
 \item les <<~sites d'escalade~>> ont-ils une existence administrative, \`a
d\'efaut de <<~l\'egale~>>~?
\item l'\'evaluation (quantitative) de la qualit\'e d'un site d'escalade donn\'e

\end{enumerate}

\section{Historique : le vocable <<~site d'escalade~>> entre dans le
vocabulaire minist\'eriel~!}

Il est int\'eressant de souligner que, pour la premi\`ere fois, les pouvoirs
publics interrogent et s'interrogent sur la qualit\'e -- donc l'existence --
des <<~sites d'escalade~>>. On ose alors esp\'erer que cette premi\`ere
\'ebauche d'action pourrait conduire \`a des incitations \'economiques et \`a
des officialisations r�glementaires, voire
l\'egales.\label{historique}Label"historique". LIGNE VIDE.

Il est int\'eressant de souligner que, pour la premi\`ere fois, les pouvoirs
publics interrogent et s'interrogent sur la qualit\'e -- donc l'existence --
des <<~sites d'escalade~>>. On ose alors esp\'erer que cette premi\`ere
\'ebauche d'action pourrait conduire \`a des
 incitations \'economiques et \`a (en plus)  LIGNE VIDE.
 
 
\begin{center}\small
\begin{tabular}{|c|c|c|c|}
\hline
Cotation & Signification & Couleur & Difficult\'e des\\
 d'ensemble & & & passages\\
\hline
E & Circuit pour enfants & Blanc & ---\\
\cline{1-1}\cline{3-3}
F & Facile & Jaune & 1--2c\\
PD& Peu difficile & Jaune & 2a--3a\\
\cline{2-3}
AD& Assez difficile & Orange & 3a--4a\\
D & Difficile & Bleu & 4a--5a\\
TD& Tr\`es difficile & Rouge & 4c--6a\\
ED& Extr\^emement difficile & Noir ou Blanc& 5c--8a\\
\hline
ABO& \multicolumn2{c|}{Pas d'ABO � Bleau}& 7c--8a\\
\hline
\end{tabular}
\end{center}

\begin{center}\small
\begin{tabular}{|c|c|c|c|}
\hline
Cotation & Signification & Couleur & Difficult\'e des\\
 d'ensemble & & & passages\\
\hline
E & Circuit pour enfants & Blanc & ---\\
\cline{1-1}\cline{3-3}
F & Facile & Jaune & 1--2c\\
PD& Peu difficile & Jaune & 2a--3a\\
\cline{2-3}
AD& Assez difficile & Orange & 3a--4a\\
D & Difficile & Bleu & 4a--5a\\
TD& Tr\`es difficile & Rouge & 4c--6a\\
ED& Extr\^emement difficile & Noir ou Blanc& 5c--8a\\
\hline
ABO& \multicolumn2{c|}{Pas d'ABO � Bleau}& 7c--8a\\
\hline D & DIFFICILE & BLEU & 4a--5a\\
\hline
\end{tabular}
\end{center}

\begin{center}\small
\begin{tabular}{|c|c|c|c|}
\hline
Cotation & Signification & Couleur & Difficult\'e des\\
 d'ensemble & & & passages\\
\hline
E & Circuit pour enfants & Blanc & ---\\
\cline{1-1}\cline{3-3}
F & Facile & Jaune & 1--2c\\
PD& Peu difficile & Jaune & 2a--3a\\
\cline{2-3}
AD& Assez difficile & Orange & 3a--4a\\
D & Difficile & Bleu & 4a--5a\\
TD& Tr\`es difficile & Rouge & 4c--6a\\
ED& Extr\^emement difficile & Noir ou Blanc& 5c--8a\\
\cline{1-1}\cline{3-3}
ABO& \multicolumn2{c|}{PAS D'ABO � BLEAU}& 7c--8a\\
\hline
\end{tabular}
\end{center}



%{\sl Correspondance entre la difficult\'e moyenne
%(cotation <<bloc>>) des circuits et leur couleur.}
%\caption
%\label{tabldiff}
%\hrule



Alors que le littoral fran\c cais, sa protection et les activit\'es qu'il
supporte, b\'en\'eficient aujourd'hui des actions d'un <<~Conservatoire~>> et
que les espaces montagnards, leur sauvegarde et l'activit\'e humaine qui s'y
d�roule, poss\`edent une l\'egislation sp\'ecifique, la <<~loi montagne~>>,
il serait int\'eressant que les pouvoirs publics s'interrogent sur la qualit\'e
d'<<~usage social~>> des {\sl sites naturels d'escalade\/}, d'ailleurs
comparables \`a de nombreux autres espaces utilis\'e pour les activit\'es de
pleine nature~: canyons, rivi\`eres, espaces souterrains...
\medskip


Ici medskip. Or le droit \`a l'<<~usage sportif~>> de ces espaces reste \`a inventer. En
effet, actuellement, le droit \`a l'usage -- et surtout \`a l'am\'enagement
pour un minimum de s\'ecurit\'e -- de ces lieux d\'epend essentiellement du
{\sl bon vouloir\/} des propri\'etaires~: propri\'etaires particuliers
\'evidemment, mais aussi propri\'etaires <<~publics~>> dans la mesure o\`u
beaucoup de ces sites sont la {\sl propri\'et\'e priv\'ee\/} des
collectivit\'es locales ou de l'\'Etat (cas des {\sl for\^ets domaniales\/}).

Et dans le cas de terrains <<~ouverts au public~>> le droit d'usage peut
souvent \^etre remis en question par le maire, {\sl premier responsable de la
s\'ecurit\'e\/} sur sa commune en fonction d'arguments parfois discutables...
sans qu'on ait pour autant l'occasion d'en discuter avec lui. Sans compter que
les arguments des escaladeurs deviennent souvent irrecevables par un maire
d\`es lors que les demandeurs ne sont ni \'electeurs ni contribuables sur sa
commune...

\bigskip Ici bigskip.
Heureusement, ce <<~bon vouloir~>> est souvent orient\'e dans un sens favorable
aux activit\'es de pleine nature, dont l'escalade, dans la mesure o\`u ces
activit\'es repr\'esentent un <<~produit touristique~>> valorisable dans la
r\'egion. Ceci est vrai dans le sud-est de la France et le Midi-Pyr\'en\'ees,
ceci l'est beaucoup moins dans le nord-ouest agricole.(voir \ref{charq}, p. \pageref{charq})
\smallskip

Ici smallskip. On se prend donc \`a esp\'erer -- alors que les <<~sites naturels d'escalade~>>
existent\footnote{Hors alpinisme et activit\'es pr\'eparatoires \`a la pratique
de l'alpinisme.} depuis plus de cinquante ans -- que les minist\`eres, en
collaboration avec les f\'ed\'erations sportives et les organismes de {\sl
protection de la nature\/}, donneront peut-\^etre un cadre institutionnel
minimum \`a des pratiques.

\section{De la charte de qualit\'e \`a l'\'evaluation de la qualit\'e}

Si le terme <<~charte de qualit\'e~>>, qui peut aussi se comprendre <<~charte
d'am\'enagement~>>, reste le plus ad�quat pour d\'esigner le pr\'esent
travail, il faut souligner que reste pos\'e le probl\`eme de
l'{\sl\'evaluation\/} de cette qualit\'e, que ce soit pour des sites existants
ou pour de nouveaux sites.
\vspace{1.5cm}

 Ici vspace 1.5cm. Th\'eoriquement,\label{charq}
 il suffit pour ce faire de prendre alin\'ea par alin\'ea la
charte de qualit\'e propos\'ee ci-apr\`es pour les <<~sites d'escalade~>>, mais
il faut ensuite {\sl pond\'erer\/} les r\'eponses donn\'ees\footnote{Pour
autant que l'on dispose d'informations valables et r\'ecentes.}. Or, s'il est
\'evident qu'une r\'eponse <<~oui~>> \`a la question sur l'ensoleillement sera
n\'egligeable face \`a une r\'eponse <<~non~>> quant \`a la fiabilit\'e de
l'\'equipement, le choix de la pond\'eration -- voire de son signe alg\'ebrique
-- est un probl\`eme de nature politique, pour lequel il n'est pas possible de
proposer m\^eme un embryon de r\'eponse... dans les d\'elais impartis. Parmi
les points sujets \`a controverse, citons \`a titre d'exemple~:

\begin{itemize}
 \item les {\sl voies de plusieurs longueurs\/} sont tr\`es pris\'ees des
grimpeurs correctement form\'es, mais un s\'erieux handicap pour les groupes
scolaires ou de d�butants~;

\item les \'equipements rapproch\'es (<<~surprotection~>>) sont peu
appr\'eci\'es, voire m\'epris\'es, par les escaladeurs de culture {\sl
alpiniste\/}, mais consid\'er\'es comme une qualit\'e pour les groupes nombreux
dont l'encadrement est limit\'e par des raisons \'economiques.

\item le cadre <<~naturel~>> est appr\'eci\'e d'une majorit\'e d'escaladeurs,
mais sera sujet \`a objection de la part des {\sl d\'efenseurs
inconditionnels\/}\footnote{On peut aussi lire~: <<~int\'egristes~>>.} de la
nature ainsi que, \`a l'autre extr\'emit\'e, de la part de ceux qui veulent
rapprocher l'escalade de la culture des banlieues <<~\`a probl\`emes~>>.

\end{itemize}

Tout ceci pour dire que l'{\sl\'evaluation\/} doit faire l'objet d'une \'etude
compl\'ementaire avant d'\^etre propos\'ee aux instances demanderesses...

\chapter{Charte de qualit\'e des sites sportifs d'escalade} Contrairement \`a
ce que beaucoup de grimpeurs et d'am\'enageurs peuvent penser, il convient
d'insister sur le fait que la qualit� globale d'un {\sl site sportif\/}
d'escalade\footnote{Il est d'ailleurs permis de s'\'etonner ici que des
minist\`eres veuillent commanditer une telle {\sl charte de qualit\'e\/} alors
que les <<~sites d'escalade~>> -- qu'ils soient <<~sportifs~>> ou
<<~d'aventure~>> n'ont aucune existence l\'egale ou administrative. Mais on
pourra se consoler en notant que les <<~pistes de ski~>> n'ont pas plus
d'existence...} n'est pas seulement fonction de de son \'equipement mais aussi
de la qualit\'e de ses acc\`es, de la propret\'e du rocher et des conditions
g\'en\'erales de pratique.

\section{Le choix du site}

\begin{itemize}
 \item Le choix du site doit \^etre tel que l'escalade puisse s'y pratiquer
commod\'ement une bonne partie de l'ann\'ee sur du rocher sec.

\item Le rocher doit \^etre solide, {\sl au moins apr\`es \'elimination des
blocs et prises instables\/}. Ceci exclut les roches fractur\'ees ou
fissur\'ees en altitude, o\`u les gels nocturnes peuvent d\'esceller des blocs
ou des prises \`a une cadence qui exc\`ede les possibilit\'es de maintenance
pr�ventive.

\item Sous r\'eserve des conditions d'acc\`es pour la maintenance et les
secours, la hauteur des rochers et des parois n'est pas un param\`etre crucial,
toutefois~:

\begin{quote}
\small
\begin{itemize}
\item Sauf dans les r\'egions pauvres en rochers, les hauteurs entre 5 et 12~m
sont � \'eviter car de telles voies sont trop courtes pour \^etre
int\'eressantes pour l'escalade <<~en t\^ete~>> et trop dangereuses pour la
pratique {\sl de bloc\/}, c'est-\`a-dire <<~en solo~>>.

\item Les hauteurs exc\'edant deux <<~longueurs~>> (soit 90~m) peuvent poser
des probl\`emes de dur\'ee, notamment en p\'eriode hivernale (journ\'ees
courtes)~; ceci implique de pr\'evoir des possibilit\'es de <<~retraite~>> ou
d'acc\`es \`a la partie sup\'erieure des voies, selon le cas.

\end{itemize}
\end{quote}

\item Le site d'escalade doit \^etre \`a l'abri des {\sl dangers objectifs\/},
c'est-\`a-dire des chutes de pierres ou de s\'eracs, des \'eboulements, des
avalanches, des l\^achers d'eau impr\'evisibles (gorges, bords de rivi\`eres)
et des d\'echarges d\'elictueuses d'objets dangereux (gravats, ordures et
voitures vol\'ees au-dessous des routes).

\item La pratique de l'escalade ne doit pas -- du moins en permanence -- \^etre
l'occasion de conflits d'usage des lieux avec d'autres int\'er\^ets
<<~l�gitimes\footnote{L\'egitimit\'e \`a \'evaluer en chaque cas, le cas
\'ech\'eant par jugement ou d\'ecision des autorit\'es l\'egales.}~>>~:

\begin{itemize}
\item chasseurs~;
\item esp\`eces animales -- notamment oiseaux -- prot\'eg\'es ou en voie de
disparition, esp\`eces v\'eg\'etales rares~;
\item nuisance inacceptable aux riverains du site~;
\item d\'eparts de voies trop pr\`es des routes~;
\item autres contraintes de d\'efense de l'environnement, notamment
d'esth\'etique dans le cas de {\sl sites class\'es\/}~;
\item absence d'autorisation -- au moins tacite -- du propri\'etaire des lieux.
\end{itemize}

\end{itemize}
 \section{Les h\'ebergements}

L'h\'ebergement ou le camping des grimpeurs ne fait pas directement partie de
la qualit\'e d'un site d'escalade. Toutefois, l'impossibilit\'e de se loger ou
de camper \`a faible distance\footnote{Sachant que les p\'eriodes o\`u l'on
grimpe ne co\"\i{}ncident pas forc\'ement avec la <<~saison touristique~>> o\`u
les campings et les g\^\i{}tes sont ouverts.} peut \^etre un s\'erieux
inconv\'enient pr\'ejudiciable \`a la qualit\'e globale d'un site.

\medskip\noindent
Ici medskip/noindent. R\'esoudre ces probl\`emes d'h\'ebergement fait donc partie de la
responsabilit\'e des am\'enageurs, et l'information correspondante {\sl doit
figurer\/} dans les topos-guides.

\section{Les acc\`es au site}

\subsection{Les modalit\'es d'acc\`es doivent \^etre acceptables par toutes les
parties}

\begin{itemize}
 \item Le chemin d'acc\`es ne doit emprunter que des chemins publics ou
autoris\'es par les propri\'etaires. Si cette autorisation n'est que tacite, il
faut disposer d'itin\'eraires de rechange au cas o\`u le propri\'etaire
changerait d'avis\footnote{Notamment en cas de vente ou d'h\'eritage.}.
\item Le point de d\'epart de la partie p\'edestre de l'acc\`es doit comporter
des possibilit\'es de stationnement en rapport avec la fr\'equentation
attendue, en des lieux ne g\^enant pas les autres usagers des lieux
(circulation automobile, troupeaux, engins agricoles, terres cultiv\'ees,
etc.). Si ce stationnement n'est pas localis\'e de mani\`ere <<~\'evidente~>>,
il doit \^etre clairement balis\'e.
\item En accord avec les autorit\'es locales, des dispositions doivent \^etre
prises, ou des dispositifs install\'es pour emp\^echer le {\sl stationnement
g\^enant\/} et la circulation des v\'ehicules (y compris 4$\times$4) sur les
chemins et routes non autoris\'es.

\end{itemize}

\subsection{Les acc\`es ne doivent pas \^etre dangereux}

\begin{itemize}
 \item L'acc\`es peut \^etre long, mais il ne doit pas pr\'esenter de dangers
objectifs (gu\'es al\'eatoires, risques d'avalanches, travers\'ees de routes
\`a grande circulation).
\item Aux p\'eriodes normales d'utilisation pour l'escalade, l'itin\'eraire
d'acc\`es ne doit pas n\'ecessiter d'autres techniques que celles de la
randonn\'ee ou de l'escalade facile. En particulier les techniques alpines de
neige ou de terrain instables sont \`a exclure, ainsi que la marche sur glacier.
\item Si la cr\'eation de l'acc\`es a n\'ecessit\'e de couper des arbustes,
ceux-ci doivent \^etre coup\'es suffisamment ras pour \'eviter que le public se
blesse gravement en cas de chute.
\item L'itin\'eraire d'acc\`es peut n\'ecessiter de franchir quelques pas
d'escalade, \`a condition que leur difficult\'e soit mod\'er\'ee (acc\`es des
secours ou de public accompagnant les grimpeurs) et qu'ils ne soient pas {\sl
dangereux\/}. Si n\'ecessaire -- et sous r\'eserve de respect de
l'environnement -- un \'equipement sp\'ecifique (c\^ables, \'echelles, cordes
ou cha\^\i{}nes fixes) peut faciliter l'acc\`es du public.

\end{itemize}

\section{Signalisation}
\begin{itemize}
 \item S'il n'est pas \'evident (acc\`es \`a vue depuis le bord de route),
l'acc\`es au site doit \^etre balis\'e pour \'eviter les acc\`es multiples et
<<~sauvages~>>. Souvent, un petit panneau indicateur peut diriger les visiteurs
vers un sentier \'evident... une fois trouv\'e le d\'epart.

\item Si un panneau plus important est install\'e, il doit proposer les
informations suivantes~:
\begin{itemize}
 \item le nom de la ou des communes~;
\item le nom et les coordonn\'ees des associations prenant en charge le site et
son \'equipement~;
\item les possibilit\'es d'appel des secours~: num\'eros de t\'el\'ephone et
situation de la cabine t\'el\'ephonique la plus proche.

\end{itemize}

\item Les panneaux et la signalisation doivent repr\'esenter un bon compromis
entre la visibilit\'e et la discr\'etion.

\item Que ce soit par un {\sl num\'ero\/} renvoyant au topo-guide, que ce soit
par l'indication peinte (caract\`eres n'exc\'edant jamais 40~mm de haut) ou par
d'autres m\'ethodes efficaces mais discr\`etes, la localisation des d\'eparts
de voies doit se faire sans ambiguit\'e, y compris pour les grimpeurs qui y
viennent pour la premi\`ere fois.

\end{itemize}
 \section{Secours}

S'il n'est pas question d'amener une route au pied et au sommet de chaque
rocher d'escalade, ni d'y installer des aires d'atterrisage d'h\'elicopt\`ere,
le sauvetage des accident\'es pr\'evisibles doit \^etre organis\'e.

\begin{itemize}
 \item Mise au courant -- et le cas \'ech\'eant formation -- des services
locaux de s\'ecurit\'e (pompiers, gendarmerie, PGHM, etc.).

\item Indication (panneaux ou topo-guides) de l'emplacement des cabines
t\'el\'ephoniques les plus proches, avec {\sl rappel du num\'ero d'appel\/} des
services de secours.

\end{itemize}

\section{\'Equipement pour l'escalade}

\subsection{Nettoyage g\'en\'eral}

Sachant que les grimpeurs passent plus de la moiti\'e de leur temps au pied des
voies, que ce soit pour assurer un coll\`egue, pour examiner la documentation
ou chercher l'itin\'eraire de leur choix, la zone situ\'ee au pied des rochers
d'escalade doit \^etre particuli\`erement prot\'eg\'ee des chutes de pierres.

Ceci implique non seulement un nettoyage des voies proprement dites, mais aussi
la purge de tout rocher susceptible de tomber sous l'effet du vent, du gel ou
d'une fr\'equentation accidentelle du secteur. Ceci implique aussi la purge de
pentes d'\'eboulis situ\'ees au-dessus des zones d'escalade et -- le cas
\'ech\'eant -- d'en interdire l'acc\`es.

\subsection{Circulation au pied des rochers}

\begin{itemize}
 \item Le sentier permettant d'acc\'eder aux voies doit \^etre tel que la
circulation des uns ne perturbe pas l'assurage des grimpeurs d\'ej\`a engag\'es
dans des voies.

\item Si le site est susceptible d'\^etre fr\'equent\'e par des groupes,
ceux-ci doivent disposer de zones de repos ou de stationnement hors des zones
de d\'epart de voies ou d'assurage.

\end{itemize}

\subsection{Nettoyage des voies}

\begin{itemize}
 \item\'elimination des blocs susceptibles de tomber spontan\'ement ou quand on
les touche, f\^ut-ce par erreur~;

\item\'elimination de la v\'eg\'etation {\sl g\^enante\/} pour l'escalade ou la
s\'ecurit\'e\footnote{Ce qui ne signifie pas d'\'eliminer {\sl toute\/}
v\'eg\'etation de la paroi. Tout ceci est une affaire de mesure et de bon
sens.}~;

\item bris pr\'eventif ou \'elimination de prises susceptibles de rompre sous
l'effort d'un grimpeur lourd ou <<~muscl\'e~>>~;

\item\'elimination de la terre aux endroits o\`u les grimpeurs sont
susceptibles de mettre leurs pieds, leurs mains ou... leur post\'erieur~;

\item les souches restantes et les moignons de branches coup\'ees ne doivent
pas pr\'esenter de danger pour les grimpeurs en cas de chute.

\end{itemize}

\subsection{Choix des voies propos\'ees}

Un site sportif peut s'adresser pr\'ef\'erentiellement aux grimpeurs de haut
niveau, aux grimpeurs <<~moyens~>> ou au d\'ebutants. Quel que soit ce choix,
il n'ob\`ere pas la qualit\'e du site. En revanche, dans la gamme de
difficult\'es choisie, il est souhaitable d'offrir divers types d'escalade,
c'est-\`a-dire pas uniquement des dalles, des di\`edres ou des surplombs.

D'autre part, malgr\'e la mode diffus\'ee par des grimpeurs excessivement
sp\'ecialis\'es dans les courtes voies <<~dures~>>, il convient de garder \`a
l'esprit que les voies de plusieurs longueurs\footnote{De difficult\'es
relativement homog\`enes...} restent appr\'eci\'ees d'un grand nombre
d'escaladeurs. Aussi -- et sous r\'eserve des probl\`emes de s\'ecurit\'e --
est-il souhaitable d'exploiter toute la hauteur de falaises mesurant plus de
40~m\`etres.

\subsection{Les points d'assurage}

\subsubsection{Les performances requises}

Texte de v\'erification de la police.
\begin{itemize}
 \item Leur r\'esistance dans la direction (approximativement) verticale doit
{\sl imp\'erativement\/} \^etre de 2500~daN (2,5 tonnes). Il va sans dire que
cette r\'esistance requise est celle de l'ensemble [rocher + scellement +
pi\`ece m\'etallique] et qu'il ne suffit pas de mettre une plaquette garantie
\`a 2500~daN sur un scellement insuffisant, d\'efectueux ou inadapt\'e \`a la
r\'esistance m\'ecanique de la roche. Ceci signifie que le choix du type
d'amarrage doit dans tous les cas se faire en fonction de la nature de la roche
et que des scellements valables par exemple dans du b\'eton <<~normalis\'e~>>
ne sont pas forc\'ement acceptables dans certaines roches tendres.
\item Dans les directions autres que celle de la chute\footnote{Compte tenu des
\'eventuels <<~renvois~>>.} la r\'esistance des points d'assurage peut \^etre
un peu plus faible. Elle ne saurait toutefois descendre au-dessous de 1500~daN.
\item Le mat\'eriau doit \^etre naturellement r\'esistant \`a la corrosion, ou
bien efficacement prot\'eg\'e contre la corrosion par un traitement de
surface\footnote{Cf. Norme UIAA.}. Ce mat\'eriau peut \^etre un acier {\sl
inox\/} de qualit\'e suffisante ou un acier galvanis\'e ou cadmi\'e~; en
revanche une simple <<~peinture~>> est insuffisante. Le niveau de protection
requis contre la corrosion d\'epend des lieux~: une protection suffisante pour
un site ensoleill\'e et sec sera sans doute inad\'equate en bord de mer ou dans
une zone d'industries chimiques polluantes.
\item Les parties \'emergentes -- notamment les <<~plaquettes d'assurage~>> --
ne doivent pas pr\'esenter d'angles dangereux et celles distantes de plus de
12~mm du rocher doivent respecter les Normes UIAA, c'est-\`a-dire pr\'esenter
au moins un rayon de courbure de plus de 10~mm.
\end{itemize}

\subsubsection{Les techniques et technologies \`a proscrire}\label{proscrire}Label"proscrire".

Il n'est pas possible, dans une <<~charte de qualit\'e~>> de donner une
liste\footnote{Mise \`a jour r\'eguli\`erement par la FFME et l'ENSA.}
exhaustive -- donc limitative -- des technologies correspondant aux
performances requises ci-dessus. En revanche, on peut passer en revue un
certain nombre de techniques malheureusement trop fr\'equentes et qui sont \`a
exclure des sites sportifs d'escalade pr\'etendant \`a offrir un \'equipement
de qualit\'e~:

\begin{itemize}
 \item les scellements par expansion ne doivent pas r\'ealiser cette expansion
en prenant appui sur le fond du forage\footnote{Cf. Norme UIAA.}. Ceci exclut
d\'efinitivement les {\sl chevilles autoforeuses\/}\footnote{D'autant plus
qu'elles se r\'ev\`elent irr\'ecup\'erables lors d'une r\'enovation ou d'une
suppression de l'\'equipement.}, quel que soit leur diam\`etre.

\item les pitons <<~classiques~>>, rarement correctement prot\'eg\'es contre la
corrosion, et dont la solidit\'e d\'ej\`a faible d\'epend en plus de
l'\'evolution de la fissure o\`u ils sont plant\'es, sous l'effet du gel ou des
mouvements de terrain.

\item les plaquettes destin\'ees \`a la sp\'el\'eologie et, d'une mani\`ere
g\'en\'erale, toutes les plaquettes dont la r\'esistance affich\'ee est soit
insuffisante, soit inexistante.

\item les goujons \`a auto-expansion de diam\`etre inf\'erieur \`a 12~mm.

\item les r\'esines polyester pour les scellements chimiques (mauvais collage
sur m\'etal et mauvaise r\'esistance \`a l'humidit\'e).

\item le scellement de pi\`eces lisses (notamment inox) \`a la r\'esine
{\sl\'epoxyacrylique\/}\footnote{Notamment C100 de Hilti.}.

\end{itemize}

\subsubsection{La localisation des points d'assurage}

Du fait de l'extr\^eme vari\'et\'e des configurations et des << client\`eles~>>
de grimpeurs, il n'existe pas de {\sl norme\/} d'espacement des points
d'assurage. Toutefois la FFME appelle\footnote{{\sl\'Equipement et
am\'enagement d'un site naturel d'escalade\/}, COSIROC/FFME, mise-\`a-jour
continuelle.} <<~protection normale~>> la situation ou l'{\sl espacement des
points d'assurage est compris entre 2,5 et 5 m\`etres ET o\`u il n'y a pas de
chute vraisemblable\footnote{Y compris par possible retour au sol apr\`es avoir
mousquetonn\'e plusieurs points d'assurage...} de plus de 4 m\`etres (pas plus
de 1,5 m\`etre si chute au sol ou sur une vire plane)\/}.

Aussi, dans le cas o\`u l'\'equipement est plus <<~engag\'e~>> ou <<
expos\'e~>> toutes dispositions doivent \^etres prises, notamment dans la
r\'edaction du topo-guide, pour que les escaladeurs ne s'engagent dans de
telles voies qu'en {\sl parfaite connaissance de cause\/}.

\subsection{Les relais}

Ils doivent offrir au moins {\sl deux\/} ancrages r\'esistant \`a 2500~daN.
Compte tenu des multiples renvois de forces, cette r\'esistance est requise
{\sl dans toutes les directions\/}. Les exigences de forme et de r\'esistance
\`a la corrosion sont les m\^emes que pour les points d'assurage ordinaires.

D'autre part, ces ancrages doivent permettre l'ancrage direct et simultan\'e --
avec les r\'esistances ci-dessus -- d'au moins {\sl quatre\/} mousquetons de
taille normale.

Si les ancrages multiples d'un relais sont reli\'es par une cha\^\i{}ne,
celle-ci doit non seulement avoir une r\'esistance d'au moins 2000~daN, mais sa
configuration doit {\sl inciter\/} les escaladeurs \`a s'amarrer aux ancrages
directs plut\^ot qu'\`a la cha\^\i{}ne, et surtout \`a ne jamais passer une
corde {\sl autour\/} de la cha\^\i{}ne.

Les exclusions relatives aux points d'assurage (voir paragraphe \ref{proscrire}, p. \pageref{proscrire})
sont {\sl a fortiori\/} valables pour les relais.

\subsection{Les points de rappel et de <<~moulinette~>>}

S'ils sont en milieu de paroi et susceptibles d'\^etre confondus avec des
ancrages de relais pour d'autres voies, existantes, futures ou envisageables,
ils doivent \^etre am\'enag\'es {\sl comme des relais\/}. Un ou plusieurs
maillons rapides\footnote{Ferm\'es, pour r\'esister \`a un choc \'eventuel.}
peuvent \^etre ajout\'es pour faciliter rappel ou moulinette.

S'ils sont en sommet de paroi\footnote{Sommet topographique et non sommet de la
zone d'usage actuel.} ils peuvent \^etre \'equip\'es de dispositifs permettant
l'installation d'une moulinette sans se d\'ecorder. La r\'esistance requise
pour ces dispositifs est du m\^eme ordre que pour les mousquetons lorsque le
doigt est ouvert, c'est-\`a-dire 700~daN au minimum. Ils doivent en outre
\^etre en mat\'eriau r\'esistant \`a l'usure par les cordes et les particules
solides qu'elles entra\^\i{}nent usuellement~; ils ne doivent donc jamais
\^etre constitu\'es d'{\sl alliage l\'eger\/} et ils doivent d'autre part faire
l'objet d'inspections p\'eriodiques\footnote{Tous les trois mois dans les zones
\`a forte fr\'equentation.}.(voir \S\ref{historique}, p. \pageref{historique})

\section{Maintenance}

Elle doit \^etre assur\'ee r\'eguli\`erement~:

\begin{itemize}
 \item pour remplacer les ancrages corrod\'es ou us\'es,

\item pour purger -- notamment \`a chaque printemps -- les blocs et prises
devenues instables du fait des \'el\'ements atmosph\'eriques,

\item pour \'eliminer la v\'eg\'etation g\^enante.

\end{itemize}

Ceci suppose en pratique que cette maintenance soit confi\'ee \`a une groupe
stable de personnes, facilement joignables si une grave anomalie est constat\'ee.

\end{document}

