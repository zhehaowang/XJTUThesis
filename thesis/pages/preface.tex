% multiple1902 <multiple1902@gmail.com>
% intro.tex
% Copyright 2011~2012, multiple1902 (Weisi Dai)
% https://code.google.com/p/xjtuthesis/
% 
% It is strongly recommended that you read documentations located at
%   http://code.google.com/p/xjtuthesis/wiki/Landing?tm=6
% in advance of your compilation if you have not read them before.
%
% This work may be distributed and/or modified under the
% conditions of the LaTeX Project Public License, either version 1.3
% of this license or (at your option) any later version.
% The latest version of this license is in
%   http://www.latex-project.org/lppl.txt
% and version 1.3 or later is part of all distributions of LaTeX
% version 2005/12/01 or later.
%
% This work has the LPPL maintenance status `maintained'.
% 
% The Current Maintainer of this work is Weisi Dai.
%

\chapter{绪论}
\echapter{Preface}
\label{PrefaceChapter}

%Sample names for dismodule and update module
%Figures Size
%Bib check
%Wording adjustments
    
\section{研究背景}
\esection{Background}
\par
当前的互联网架构是基于70年代提出的TCP/IP协议栈实现的,其中由IP地址标识的目的端点、源端点地址是上层通信的基础。然而随着硬件技术和网络应用的发展,网络变得更加面向内容,而不是面向端点。人们更加在意网络所提供的内容,而不是请求的内容在哪个IP地址。内容请求和发布成为了网络通信的主要组成部分。
\par
命名数据网(NDN, Named data networking)利用这一点,摒弃TCP/IP协议栈对地址的需求,将所有上层通信归纳为由数据名称标识的数据请求和数据应答间的交互。这样的设计思路带来的优势有对网络多播和数据共享的天然支持,对多个连接的同时利用,和面向内容的信任模型等等。然而同样也带来了许多挑战,比如应用层的各个应用需要根据其需求,进行命名空间的重新设计。
\par
网络游戏是网络的重要应用之一。基于IP的网络游戏经历了由点对点(P2P, peer-to-peer)架构,到客户端服务器(Client/Server, C/S)架构,再到P2P重新开始被研究的发展历程。C/S架构具有着控制方便的优势,但是有着流量汇聚在几个服务器,和服务器单点出现异常会直接导致服务不可使用的劣势。然而由于P2P架构带来的控制上的困难性,当前流行的商业化的网络游戏一致采用了C/S架构。
\par
本文提供基于NDN的大型网络游戏(Massive Multiplayer Online Role Playing Game, MMORPG)应用的设计。其利用了NDN网络的优点,并且是自组织、采用纯P2P架构的。该应用不仅是网络游戏联机模型的新思路,而且对别的需要基于位置进行发现、同步的NDN应用有启发意义。同时,该应用是基于虚拟位置,进行发现、同步,并基于NDN的第一款应用。
\section{研究现状}
\esection{CurrentStatus}

\subsection{新网络架构}
\par
新网络架构是十年来的新兴的研究领域,旨在解决当前TCP/IP协议栈出现的无法满足用户需求的许多问题。当前由美国自然科学基金会(NSF)资助的未来网络结构\footnote{公布于NSF的站点,http://www.nets-fia.net/}(Future Internet Architecture)研究项目主要有NDN项目,MobilityFirst项目,NEBULA项目,eXpressive Internet Architecture项目,ChoiceNet项目。
\par
这些项目的共同特点是认识到了TCP/IP模型不再适合当前的网络使用模型。各项目在针对的点上并不相同,除了本文重点描述的NDN项目外,NEBULA项目利用大规模的云存储技术,使得服务提供、控制变得更加迅速和方便;MobilityFirst项目通过网络资源利用的优化,实现流动性和效率的平衡。
\subsection{新网络架构的主要方向}
\setcounter{subsubsection}{0}
\par
新网络架构的研究主要有以下方向,分别试图解决TCP/IP面临的不同问题。
\subsubsection{面向数据的研究方向}
ICN(Information-Centric Networking):基于内容的网络架构。NDN属于ICN的范畴,其是一种完全摒弃了TCP/IP的革命性方案。该方案为流动性、安全等当前面临的传统问题提供了一套新的解决思路。
面向流动性和网络普及的研究方向。在思路上将流动性(Mobility)作为常态,而不是网络中出现的异常。从而为自组织、高流动性的临时网及设备,和多连接性的研究提供了新的思路。
\subsubsection{面向云计算的研究方向}
将计算和存储集中在远程的一组节点,从而使得资源、服务的共享更加高效和自然。同时带来了安全、信任、鲁棒性、可拓展性等数个问题的新的挑战。
\subsubsection{网络安全研究方向}
安全相关的协议并不包含在IP的基础设计中,而是作为特定网络层、传输层或应用层协议的附加层。由于安全问题逐渐成为了网络架构需要考虑的基础问题,IP这样的设计思路可能不再适合当今的用例。
\subsubsection{控制面、物理面分层的研究方向}
SDN(Software Defined Networking)是该方向的代表。将网络的控制面和物理面分开,对两个模块分别进行设计、优化,从而提高网络的灵活性。
\subsection{NDN项目}
\par
NDN项目将数据及其名称,而不是网络地址,作为网络服务的最基本对象。由此带来了网络层及其上层协议的重新设计。
NDN的研究始于2006年,2009年Van Jacobson将其初步实现命名为CCNx\cite{NDNRefOriginal}。PARC的研究小组提供了CCNx的实现\footnote{CCNx项目的实现及介绍,http://www.ccnx.org/。}:基于内容的网络架构(Content Centric Networking(CCN,CCNx))。
\par
在此之后,当今项目的理念仍然与最初的设计基本一致,而实现上已有了许多完善。在2014年,NDN小组完成了路由节点的新实现,NRD/NFD(NDN Routing Daemon/NDN Forwarding Daemon)和NDN-tlv,基于类型-长度-值(tlv)的NDN数据包结构;在2013年,完成了ChronoSync同步模型的设计、实现,完成了客户端函数库的面向对象实现等等。
\par
目前作者的指导老师,美国加州大学洛杉矶分校的Prof. Lixia Zhang 和Prof. Jeff Burke在领导项目的研究。
\subsection{游戏联机}
\setcounter{subsubsection}{0}
\par
随着网络愈发便捷,多人联机网络游戏成为网络娱乐的重要组成部分。最为流行的MMORPG,魔兽世界,有着峰值超过1200万的注册用户\footnote{数据来源,http://www.statista.com/statistics/276601/number-of-world-of-warcraft-subscribers-by-quarter/};而其开发公司,暴雪公司有着8000台服务器为该游戏提供联机支持。随着许多MMORPG的商业化,如何更廉价而高效的提供网络游戏联机服务也成为了网络研究的一个方向。近来,多篇基于IP网络架构的P2P游戏联机方案(ref)被多个研究组织提出。但是这些方案都基于TCP/IP协议栈,其P2P基本通过分布式哈希表(Distributed Hash Table)的部署实现。在\ref{ComparisonSection}部分,将对部分文章和思路做出更详细的总结和对比。
\section{研究目的和意义}
\par
本研究项目的主要目的在于以下几点。
\subsubsection{开发基于NDN的MMORPG应用,为更高效的网络游戏通信提供实现;}
\subsubsection{解决完全自组织带来的发现、同步及控制问题;}
\subsubsection{衡量NDN网络架构在类似用例下的实用性;为同类型项目提供命名空间开发的启发和样例。}
\par
项目的意义主要有:通过基于P2P架构的自组织设计,解决当前大型网络联机游戏面临的几个问题,即服务器负载过重,维护成本高昂,依然无法满足超大量用户同时存在于一个虚拟空间的需求。衡量NDN的设计思路是否可以为此类应用带来效率上的提升。为类似应用,例如ICN-CarSpeak\cite{V2VUseCaseRef}的开发提供启发和帮助。
\section{研究内容}
\par
本文研究的内容涵盖多个层面的设计和实现。设计角度而言, 本文分析了NDN的已有同步模型,并根据MMORPG的用例需求进行了新的同步方案的设计和改良。同时,本文分析了同步过程面临的抽象问题kNN问题的主要设计思路及实现。实现角度而言,为了对基于Unity游戏引擎的游戏实现提供支持,作者开发了基于C\#语言的NDN客户端函数库;并基于自主开发的函数库,提供了发现模块和更新模块的实现;最后在游戏引擎中进行了应用和测试。
\section{论文结构}
\par
本文的组织结构如下:
第\ref{PrefaceChapter}章:绪论。本章介绍论文研究背景、研究现状、研究的目的和意义,研究内容以及总体的组织结构。
第\ref{RelatedChapter}章:相关技术概述。 
第\ref{AnalysisChapter}章:系统需求分析。本章对系统的需求、系统框架、系统功能以及数据流进行了详细的描述和分析。
第\ref{DesignChapter}章:系统设计。本章主要对系统使用到的类进行了详尽的分析和描述。 
第\ref{ImplementationChapter}章:系统实现。本章详细讲述了系统实现的过程、以及系统测试。
第\ref{EvaluationChapter}章:系统分析及优化。本章分析系统是否达到了理论要求,并提出优化方案,和其它的设计方案做出对比。
第\ref{ConclusionChapter}章:总结与展望。对系统进行全面的归纳和总结,对尚存在的问题进行分析,并对进一步工作进行展望和规划。

